\documentclass [15pt,titlepage]{jarticle}
\usepackage[italian]{babel} 
\usepackage[utf8]{inputenc}
\usepackage[T1]{fontenc} %chess package 
\usepackage{xskak} 
\usepackage{texmate}



\usepackage{rsfso,graphics,comment}
\usepackage{amsmath,amssymb,multicol}
\usepackage{amsthm}
\usepackage{color,multicol}
\usepackage{skak,xskak,chessboard}
\usepackage[hypertex]{hyperref}
\usepackage{pifont}
%%%%%%%%%%%%%%%%%%%%%%%%%%%%%%%%%%%%%%%%%%%%%%%%%%%%%%%%%%%%%%%%%%%%%%%%%%%%%%
\begin{comment}
楕円曲線論入門
J.H.シルヴァーマン/J.テイト 著
\end{comment}
%%%%%%%%%%%%%%%%%%%%%%%%%%%%%%%%%%%%%%%%%%%%%%%%%%%%%%%%%%%%%%%%%%%%%%%%%%%%%%
\begin{document}


%%%%%%%%%%%%%%%%%%%%%%%%%%%%%%%%%%%%%%%%%%%%%%%%%%%%%%%%%%%%%%%%%%%%%%%%%%%%%%
\begin{comment}
演習問題(1.6)
\end{comment}
%%%%%%%%%%%%%%%%%%%%%%%%%%%%%%%%%%%%%%%%%%%%%%%%%%%%%%%%%%%%%%%%%%%%%%%%%%%%%%
\textbf{1.6.}
\textbf{(a)}
最初に分子を$t$で表す。
\begin{align*}
	a+b\cos\theta+c\sin\theta 
		&=
\displaystyle	
	a+b(2\cos^2\frac{\theta}{2}-1)+2c\sin\frac{\theta}{2}\cos\frac{\theta}{2}\\
		&=
	\left(
	\frac{a-b}{\cos^2\frac{\theta}{2}}+2c\cdot\frac{\sin\frac{\theta}{2}}{\cos\frac{\theta}{2}}+2b
	\right)
	\cdot\cos^2\frac{\theta}{2}\\
		&=(a-b)+2c\cdot\frac{t}{1+t^2}+2b\cdot\frac{1}{1+t^2}\\
		&=
	\left\{
		(a-b)(1+t^2)+2c\cdot t+2b
	\right\}\frac{1}{1+t^2}\\
		&=
	\left\{
		(a-b)t^2+2c\cdot t+a+b
	\right\}\frac{1}{1+t^2}
\end{align*}
次にこれを用いてもとの積分を$t$で表す。
\begin{eqnarray*}
\displaystyle
	\int\frac{a+b\cos\theta+c\sin\theta}
			{d+e\cos\theta+f\sin\theta}d\theta
	&=&
	\int\frac{a+b(2\cos^2\frac{\theta}{2}-1)+2c\sin\frac{\theta}{2}\cos\frac{\theta}{2}}
			{d+e(2\cos^2\frac{\theta}{2}-1)+2f\sin\frac{\theta}{2}\cos\frac{\theta}{2}}d\theta \\
	&=&
	\int\frac{(a-b)t^2+2c\cdot t+a+b}{(d-e)t^2+2f\cdot t+d+e}\cdot \frac{2}{1+t^2}dt
\end{eqnarray*}


\newpage

\textbf{(b)}計算をする前に2つの公式の紹介と導出をする。\\
($x,t$のどちらか一方で表そうとしたが疲れた。)

\begin{eqnarray}
\int x\log xdx
	&=&\frac{1}{2}x^2\log x-\frac{1}{4}x^2 \\
\int\frac{t^2}{1+t}dt
	&=&\log\lvert1+t\rvert-\frac{1}{2}(t-3)(t+1)
\end{eqnarray}

(1)の式を求める。
\begin{align*}
\int x\log xdx
	&=x(x\log x-x)-\int (x\log x-x)dx \\
2\int x\log xdx
	&=x(x\log x-x)+\frac{1}{2}x^2 \\
\int x\log xdx
	&=\frac{1}{2}x^2\log x-\frac{1}{4}x^2 \\
\end{align*}

(2)の式を求める。
\begin{align*}
\int\frac{t^2}{1+t}dt
	&=
\int t^2\left(\log\lvert1+t\rvert\right)^\prime dt	\\
	&=
t^2\log\lvert1+t\rvert-\int 2t\log\lvert1+t\rvert dt\\
	&=
t^2\log\lvert1+t\rvert-
	2\int 
	\left\{(1+t)\log\lvert1+t\rvert -\log\lvert1+t\rvert \right\}dt\\
	&=
t^2\log\lvert1+t\rvert
-\left\{
	(1+t)^2\log \lvert1+t\rvert-\frac{1}{2}(1+t)^2
\right\}
+2\left\{(1+t)\log\lvert1+t\rvert-(1+t)\right\}\\
	&=\log\lvert1+t\rvert-\frac{1}{2}(t-3)(t+1)
\end{align*}

\begin{align*}
\int \{
	(t-1)+\frac{1}{1+t}
	\}
\end{align*}
で計算すれば楽になると後から気づいた。定数項は各々が好きにするといい。\\
%mathematicaの計算と(2)の公式が一致したので今回の計算を採用した。

\newpage
最初に分母を$t$で表す。
(\textbf{(a)}で求めた分子の式に$a=1,b=1,c=1$を
代入すればいい事に見直し気づいたが、せっかくなので残した。)
\begin{align*}
1+\cos x+\sin x 
	&=1+(2\cos^2\frac{x}{2}-1)+2\cos\frac{x}{2}\sin\frac{x}{2}\\
	&=\frac{(1+t^2)+(1-t^2)+2t}{1+t^2}\\
	&=2\cdot\frac{1+t}{1+t^2}
\end{align*}

\textbf{(a)}で求めた分子の式と先ほどの公式(2)を用いる。

\begin{eqnarray*}
\displaystyle
	\int\frac{a+b\cos\theta+c\sin\theta}
			{1+\cos\theta+\sin\theta}d\theta
	&=&
	\int\{(a-b)t^2+2c\cdot t+(a+b)\}\cdot\frac{1+t^2}{2(1+t)}\cdot\frac{2}{1+t^2}dt\\
	&=&
	\int
	\left\{(a-b)\cdot\frac{t^2}{(1+t)}+2c\cdot\frac{t}{(1+t)}+(a+b)\cdot\frac{1}{(1+t)}\right\}dt\\
	&=&
	(a-b)\left\{\log\lvert1+t\rvert-\frac{1}{2}(t-3)(t+1)\right\}\\
	& &+2c\left(t-\log\lvert1+t\rvert\right)+(a+b)\log\lvert1+t\rvert\\
	&=&
2(a-c)\log\lvert1+t\rvert
-\frac{1}{2}(a-b)(t-3)(t+1)+2c\cdot t
\end{eqnarray*}



%%%%%%%%%%%%%%%%%%%%%%%%%%%%%%%%%%%%%%%%%%%%%%%%%%%%%%%%%%%%%%%%%%%%%%%%%%%%%%
\begin{comment}
演習問題(1.13)
\end{comment}
%%%%%%%%%%%%%%%%%%%%%%%%%%%%%%%%%%%%%%%%%%%%%%%%%%%%%%%%%%%%%%%%%%%%%%%%%%%%%%
\textbf{1.13.}
\textbf{(a)}
\begin{align*}
	\alpha	&=u^3+v^3						\\
			&=(u+v)\{(u+v)^2-3uv\}		\tag{1}	\\
			&=(u+v)\{(u-v)^2+uv\}		\tag{2}				
\end{align*}
 (1),(2)から$uv$の項を除去する。すると次の式が得られる。
\begin{equation*}
	4\alpha=(u+v)\{(u+v)^2+3(u-v)^2\}
\end{equation*}
式変形をして
\begin{align*}
	4\alpha		&=(u+v)\{(u+v)^2+3(u-v)^2\}\\
	\frac{4\alpha}{(u+v)^3}
				&=1+3\frac{(u-v)^2}{(u+v)^2}
\end{align*}
 4と3の最小公倍数が12であるから12の倍数であたりをつけた。\\
あとは素因数分解で両辺の2と3の指数を合わせる。
\begin{align*}			
	4\alpha
	\left(\frac{x}{12\alpha}\right)^3		
				&=1+3\left(\frac{y}{36\alpha}\right)^2	\\
	 y^2			&=x^3-432\alpha^2			
\end{align*}

\newpage
%%%%%%%%%%%%%%%%%%%%%%%%%%%%%%%%%%%%%%%%%%%%%%%%%%%%%%%%%%%%%%%%%%%%%%%%%%%%%%
\begin{comment}
演習問題(1.16)
\end{comment}
%%%%%%%%%%%%%%%%%%%%%%%%%%%%%%%%%%%%%%%%%%%%%%%%%%%%%%%%%%%%%%%%%%%%%%%%%%%%%%
\textbf{1.16.}
\textbf{(a)}
\begin{align*}
L
	&=4\int^\frac{\pi}{2}_0
		\sqrt{
			(\{\alpha\sin\theta)^{\prime}\}^2+\{(\beta\cos\theta)^{\prime}\}^2
			}\cdot d\theta\\
	&=4\int^\frac{\pi}{2}_0
		\sqrt{
			(\alpha\cos\theta)^2+(\beta\sin\theta)^2
			}\cdot d\theta\\
	&=4\alpha\int^\frac{\pi}{2}_0\sqrt{
			\cos^2\theta+\left(\frac{\beta}{\alpha}\right)^2\sin^2\theta
				}\cdot d\theta\\
	&=4\alpha\int^\frac{\pi}{2}_0\sqrt{1-\left(1-\frac{\beta^2}{\alpha^2}\right)\sin^2\theta
				}\cdot d\theta\\
	&=4\alpha\int^\frac{\pi}{2}_0\sqrt{1-k^2\sin^2\theta}\cdot d\theta\\
\end{align*}


\textbf{(b)
$\alpha=\beta$}
\begin{align*}
\displaystyle
L
	&=4\alpha\int^\frac{\pi}{2}_0 d\theta\\
	&=4\alpha\left[\frac{\pi}{2}-0\right]\\
	&=2\alpha\pi
\end{align*}


\textbf{(c)}
\begin{align*}
\displaystyle
4\alpha\int^1_0\sqrt{\frac{1-k^2t^2}{1-t^2}}dt=4\alpha\int^1_0\frac{1-k^2t^2}{\sqrt{(1-t^2)(1-k^2t^2)}}dt
\end{align*}
 

\end{document}